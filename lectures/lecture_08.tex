\section{Ключевые статистики и тесты в теории принятия решений. Дискретное 
время}
\subsection{Лемма Неймана-Пирсона}
Наша ближайшая цель состоит в том, чтобы естественным образом подойти к 
описанию тех <<достаточных>> статистик от наблюдаемых данных, на основании 
которых принимаются <<оптимальные>> решения.

Начнём с задачи различения двух гипотез. Пусть \((\xi_{1}, \xi_{2} \ldots, 
\xi_{n}, \ldots)\)~--- независимые и одинаково распределённые случайные 
величины. Над ними было проведено наблюдение и в результате получилась 
реализация \((x_{1}, x_{2}, \ldots, x_{n}, \ldots)\). Но нужно уточнить, что 
понимать в данном случае под <<независимостью и одинаковой распределённостью>>.

Пусть \((\Omega, \F)\)~--- измеримое пространство (пространство элементарных 
исходов и сигма-алгебра событий). На нём вводится две вероятностные меры 
\(\Pr_{0}\) и \(\Pr_{\infty}\). Предположение независимости случайных величин 
\((\xi_{1}, \xi_{2} \ldots, \xi_{n}, \ldots)\) означает независимость 
относительно обеих мер, то есть для любого \(n \in \N\) и любых \(A_{1}, 
\ldots, A_{n} \in \B(\R)\) (будем считать, что мы смотрим на одномерные 
случайные величины)
\[
	\Pr_{\theta}(\xi_{1} \in A_{1}, \ldots, \xi_{n} \in A_{n}) = 
	\Pr_{\theta}(\xi_{1} \in A_{1}) \cdot \ldots \cdot \Pr_{\theta}(\xi_{n} \in 
	A_{n}), \text{ где } \theta = 0 \text{ или } \infty.
\]

Далее мы будем считать, что случайные величины \(\xi_{k}\) имеют функции 
распределения \(F = F_{\theta}(x) (= \Pr_{\theta}(\xi_{k} \leq x))\), у которых 
есть плотность \(f = f_{\theta}(x)\):
\[
	\diff F_{\theta}(x) = f_{\theta}(x)\mu(\diff x),
\]
где \(\mu\)~--- некоторая (\(\sigma\)-конечная) мера. В качестве такой меры 
всегда можно взять \(\mu(\diff x) = (\Pr_{0}(\diff x) + \Pr_{\infty}(\diff 
x))/2\). Впрочем, часто будем полагать, что \(\mu(\diff x) = \diff x\), то есть 
\(\mu\)~--- это мера Лебега (в этом случае говорят, что функция распределения 
\emph{абсолютно непрерывна}).

Из независимости и одинаковой распределённости следует, что плотность 
\(p_{\theta}(x_{1}, \ldots, x_{n})\) совместного распределения \(F_{\theta} 
(x_{1}, \ldots, x_{n}) = \Pr_{\theta}(\xi_{1} \leq x_{1}, \ldots, \xi_{n} \leq 
x_{n})\) равна%
\footnote{В дискретном случае полагайте, что \(f_{\theta}(x) = 
\Pr_{\theta}(\xi_{1} = x)\).}
\begin{equation}\label{eq:density-independent}
	p_{\theta}(x_{1}, \ldots, x_{n}) = f_{\theta}(x_{1}) \ldots 
	f_{\theta}(x_{n})
\end{equation} 

Одну из главных ролей будет играть \emph{отношение правдоподобия}, которое 
вводилось в курсе математической статистики:
\begin{equation}
	L_{n} = \frac{f_{0}(x_{1}) f_{0}(x_{2})\ldots f_{0}(x_{n})} 
	{f_{\infty}(x_{1}) f_{\infty}(x_{2}) \ldots f_{\infty}(x_{n})}.
\end{equation}

Как вам известно, оно проявляется в задаче различения двух простых гипотез 
\(H_{0}\) и \(H_{\infty}\) о том, какую плотность, \(f_{0}\) или же 
\(f_{\infty}\), имеют наблюдаемые случайные величины \(\xi_{1}, \ldots, 
\xi_{N}\). Решение этой задачи даёт \emph{лемма Неймана-Пирсона}, один из 
вариантов которой состоит в следующем.

Пусть есть \(N\) наблюдений \(x_{1}, \ldots, x_{N}\) над случайными величинами 
\(\xi_{1}, \ldots, \xi_{N}\). По этим наблюдениям нужно сделать вывод, какая из 
гипотез~--- \(H_{0}\) (\(\theta = 0\)) или же \(H_{\infty}\) (\(\theta = 
\infty\))~--- имеет место. Будем предполагать, что соответствующими плотностями 
функций распределения \(F_{\theta}(x_{1}, \ldots, x_{N})\) являются 
\(p_{\theta}(x_{1}, \ldots, x_{N})\):
\[
	F_{\theta}(x_{1}, \ldots, x_{N}) = \int\limits_{-\infty}^{x_{1}} \cdots 
	\int\limits_{-\infty}^{x_{N}} p_{\theta}(y_{1}, \ldots, y_{N})\mu(\diff 
	y_{1}, \ldots, \diff y_{N}),
\]
где \(\mu\)~--- некоторая \(\sigma\)-конечная мера на \(\B(\R)\). В случае, 
когда \(\xi_{k}\) независимы и одинаково распределены плотности \(p_{\theta}\) 
(по мере \(\mu(\diff y_{1}, \ldots, \diff y_{N}) = \mu(\diff y_{1}) \ldots 
\mu(\diff y_{N})\)) задаются формулой \eqref{eq:density-independent}.

С задачей различения двух гипотез связано одно определение:
\begin{definition}
	Всякую (измеримую) функцию \(d(x_{1}, \ldots, x_{N})\), принимающую два 
	значения: \(\mathbb{H}_{0}\) (верна гипотеза \(H_{0}\)) и 
	\(\mathbb{H}_{\infty}\) (верна гипотеза \(H_{\infty}\)), будем называть 
	\emph{решающей функцией}.
\end{definition}
Наряду с решающей функцией в статистике вводятся понятия ошибок первого и 
второго рода.
\begin{definition}
	Пусть \(d(x_{1}, \ldots, x_{N})\)~--- решающая функция. Вероятностью 
	\emph{ошибки первого рода} \(\alpha(d)\) будем называть \(\alpha(d) = 
	\Pr{\text{приняли } H_{0} \given \text{верна } H_{\infty}}\). Аналогично, 
	вероятность \emph{ошибки второго рода} \(\beta(d)\) равна \(\beta(d) = 
	\Pr{\text{приняли } H_{\infty} \given \text{верна } H_{0}}\).
	% какая-то херня происходит с обозначениями
\end{definition}

Как же выбрать <<оптимальное>> решающее правило \(d\)? В данном случае не 
понятно, что вкладывается в слово <<оптимальное>>~--- больно много трактовок. 
Будем считать, что в нашем случае <<оптимальность>> означает следующее: 
решающее правило \(d^{*}\) считается оптимальным, если сумма вероятностей 
ошибок первого и второго рода минимальна:
\begin{equation}
	\alpha(d^{*}) + \beta(d^{*}) = \inf\limits_{d}[\alpha(d) + \beta(d)] 
	\overset{\bigtriangleup}{=} \mathrm{Er}(N; H_{0}, H_{\infty}).
\end{equation}

\begin{leftbar}
\begin{small}
	Стоит заметить, что \(\mathrm{Er}(N; H_{0}, H_{\infty})\) можно посчитать 
	по следующей формуле:
	\[
		\mathrm{Er}(N; H_{0}, H_{\infty}) = 1 - \frac{1}{2}\|\Pr_{0}^{(N)} - 
		\Pr_{\infty}^{(N)}\|,
	\]
	где \(\Pr_{\theta}^{(N)}(\diff x_{1}, \ldots, \diff x_{N}) = 
	f_{\theta}(x_{1}) \ldots f_{\theta}(x_{N})\diff x_{1} \ldots \diff x_{N}\), 
	а \(\| \cdot \|\) есть вариация меры (со знаком): 
	\[
		\|Q\| \overset{\bigtriangleup}{=} 2\sup_{A} |Q(A)|.
	\]
	
	Из этого можно сделать следующий неформальный вывод: если меры 
	\(\Pr_{0}^{(N)}\) и \(\Pr_{\infty}^{(N)}\) <<сидят>> на разных множествах, 
	то \(\|\Pr_{0}^{(N)} - \Pr_{\infty}^{(N)}\| = 2\) и \(\mathrm{Er}(N; H_{0}, 
	H_{\infty}) = 0\). Это означает, что возможно безошибочное разделение 
	гипотез. Если же меры близки, то то \(\|\Pr_{0}^{(N)} - 
	\Pr_{\infty}^{(N)}\| \sim 0\) и \(\mathrm{Er}(N; H_{0}, H_{\infty}) \sim 
	1\). 
\end{small}
\end{leftbar}

Есть ещё одна формулировка, которую обычно называют \emph{условной}. Пусть 
\(D_{a} = \{d: \alpha(d) \leq a\}\)~--- множество решающих правил с 
вероятностью ошибки первого рода не больше \(a\). Требуется найти \(d_{a}^{*}\) 
из \(D_{a}\) (если такое существует), что оно минимизирует вероятность ошибки 
второго рода:
\begin{equation}
	\beta(d_{a}^{*}) = \inf\limits_{d \in D_{a}} \beta(d).
\end{equation}

Теперь уместно рассказать о \emph{рандомизированных} решающих правилах. Пусть 
\(\phi = \phi(x_{1}, \ldots, x_{N})\) принимает значения в \([0, 1]\). Будем 
интерпретировать \(\phi(x_{1}, \ldots, x_{N})\), как вероятность принять 
гипотезу \(H_{0}\), если были получены наблюдения \(x_{1}, \ldots, x_{N}\) над 
случайными величинами \(\xi_{1}, \ldots, \xi_{N}\). Теперь введём два 
обозначения, считая, что \(\E_{\theta}\) есть матожидание, взятое по 
вероятностной мере 
\(\Pr_{\theta}\),
\begin{align*}
	\alpha(\phi) = \E_{\infty}[\phi(\xi_{1}, \ldots, \xi_{N})], \quad 
	\beta(\phi) = \E_{0}[1 - \phi(\xi_{1}, \ldots, \xi_{N})].
\end{align*}
Несложно понять, что это есть ни что иное, как вероятности ошибок первого и 
второго рода соответственно.

Оптимальность в данном случае вводится почти так же, как в условной 
формулировке. Пусть \(\Phi_{a} = \{\phi : \alpha(\phi) \leq a\}\)~--- множество 
рандомизированных критериев с вероятностью ошибки первого рода не больше \(a\). 
Тогда решающая функция \(\phi_{a}^{*}\) будет называться оптимальным 
(рандомизированным) тестом, если 
\begin{equation}\label{eq:optimal-random-test}
	\beta(\phi_{a}^{*}) = \inf\limits_{\phi \in \Phi_{a}} \beta(\phi).
\end{equation}

А теперь можно дать лемму Неймана-Пирсона:
\begin{lemma}[Нейман, Пирсон]
	Для любого \(a \in [0, 1]\) найдутся такие константы \(\lambda_{a}^{*}\) и 
	\(h_{a}^{*}\), что рандомизированный критерий
	\begin{equation}\label{eq:neyman-pearson-criterion}
		\phi^{*}(x_{1}, \ldots, x_{N}) = 
		\begin{cases}
			1, & p_{0}(x_{1}, \ldots, x_{N}) > h_{a}^{*}p_{\infty}(x_{1}, 
			\ldots, x_{N}), \\
			\lambda_{a}^{*}, & p_{0}(x_{1}, \ldots, x_{N}) = 
			h_{a}^{*}p_{\infty}(x_{1}, \ldots, x_{N}), \\
			0, & p_{0}(x_{1}, \ldots, x_{N}) < h_{a}^{*}p_{\infty}(x_{1}, 
			\ldots, x_{N}) \\
		\end{cases}
	\end{equation}
	является оптимальным в классе \(\Phi_{a}\).
\end{lemma}

Из \eqref{eq:neyman-pearson-criterion} понятно, что ключевая статистика в лемме 
Неймана-Пирсона~--- это отношение правдоподобия (если знаменатель не обращается 
в ноль, конечно):
\begin{equation}\label{eq:likelihood-ratio}
	L_{N} = \frac{p_{0}(x_{1}, \ldots, x_{N})}{p_{\infty}(x_{1}, \ldots, 
	x_{N})}.
\end{equation}

Казалось бы, зачем вводятся рандомизированные тесты? Оказывается, что инфинум в 
\eqref{eq:optimal-random-test} достигается на тесте, для которого вероятность 
ошибки первого рода \emph{в точности} равна \(a\). Этого, в общем случае, 
нельзя достичь без рандомизированных критериев, что будет \emph{явно} 
использоваться при доказательстве леммы.
\begin{proof}
	Для начала предположим, что мы нашли \(\lambda_{a}^{*}\) и \(h_{a}^{*}\) 
	для критерия \eqref{eq:neyman-pearson-criterion} такие, что 
	\(\E_{\infty}[\phi^{*}(\xi_{1}, \ldots, \xi_{N})] = a\). Докажем, что для 
	любого другого критерия \(\phi\) из класса \(\Phi_{a}\) вероятность ошибки 
	второго рода не меньше:
	\begin{equation}\label{eq:neyman-pearson-error-inequality}
		\beta(\phi^{*}) = \E_{0}[1 - \phi^{*}(\xi_{1}, \ldots, \xi_{N})] \leq 
		\E_{0}[1 - \phi(\xi_{1}, \ldots, \xi_{N})] = \beta(\phi).
	\end{equation}
	Это, в свою очередь, равносильно тому, что 
	\[
		\E_{0}[\phi^{*}(\xi_{1}, \ldots, \xi_{N})] \geq \E_{0}[\phi(\xi_{1}, 
		\ldots, \xi_{N})].
	\]
	
	Распишем разность матожиданий:
	\[
		\E_{0}[\phi^{*}(\xi_{1}, \ldots, \xi_{N}) - \phi(\xi_{1}, 
		\ldots, \xi_{N})] = \int\limits_{\R^{N}} (\phi^{*}(\mathbf{x}) - 
		\phi(\mathbf{x}))p_{0}(\mathbf{x})\mu(\diff \mathbf{x}),
	\]
	где \(\mu(\diff \mathbf{x}) = \mu(\diff x_{1}, \ldots, \diff x_{N})\).
	
	Теперь сделаем детур и докажем, что
	\begin{equation}\label{neyman-pearson-inequality-2}
		\int\limits_{\R^{N}} (\phi^{*}(\mathbf{x}) - \phi(\mathbf{x})) 
		(p_{0}(\mathbf{x}) - h_{a}^{*}p_{\infty}(\mathbf{x}))\mu(\diff 
		\mathbf{x}) \geq 0.
	\end{equation}
	
	Действительно, разобъём его на два (для компактности опустим аргументы 
	функций):
	\begin{align*}
		\int\limits_{\R^{N}} (\phi^{*} - \phi)(p_{0} - h_{a}^{*}p_{\infty}) 
		\mu(\diff \mathbf{x}) &= \int\limits_{\{\mathbf{x} \in 
		\R^{N}\,:\,\phi^{*} > \phi\}} (\phi^{*} - \phi)(p_{0} - 
		h_{a}^{*}p_{\infty}) \mu(\diff \mathbf{x}) + \\
		&+ \int\limits_{\{\mathbf{x} \in \R^{N}\,:\,\phi^{*} < \phi\}} 
		(\phi^{*} - \phi)(p_{0} - h_{a}^{*}p_{\infty}) \mu(\diff \mathbf{x}).
	\end{align*}
	
	Мы можем сказать, что на множестве \(\{\mathbf{x} \in \R^{N} : \phi^{*} > 
	\phi\}\) критерий не обращается в ноль: \(\phi^{*} > 0\) (так как \(\phi\) 
	принимает значения в \([0, 1]\)). Если \(\phi^{*} > 0\), то из определения 
	критерия получаем, что \(p_{0} \geq h_{a}^{*}p_{\infty}\). Отсюда получаем, 
	что первый интеграл неотрицателен.
	
	Аналогично, мы можем сказать, что на множестве \(\{\mathbf{x} \in 
	\R^{N}\,:\,\phi^{*} < \phi\}\) критерий не обращается в единицу: \(\phi^{*} 
	< 1\). Следовательно, \(p_{0} \leq h_{a}^{*}p_{\infty}\) и второй интеграл 
	тоже неотрицателен. Комбинируя эти результаты, получаем утверждение 
	\eqref{neyman-pearson-inequality-2}. Из него сразу же получается 
	\eqref{eq:neyman-pearson-error-inequality}:
	\begin{align}
		\int\limits_{\R^{N}} (\phi^{*} - \phi)p_{0}\mu(\diff \mathbf{x}) \geq 
		h_{a}^{*} \int\limits_{\R^{N}} (\phi^{*} - \phi)p_{\infty}\mu(\diff 
		\mathbf{x}) = h_{a}^{*}(\E_{\infty}[\phi^{*}] - \E_{\infty}[\phi]) \geq 
		0.
	\end{align}
	
	Рассуждение выше опиралось на то, что существуют такие \(\lambda_{a}^{*}\) 
	и \(h_{a}^{*}\), что вероятность ошибки первого рода равна \(a\): 
	\(\E_{\infty}[\phi^{*}(\xi_{1}, \ldots, \xi_{N})] = a\). Теперь докажем, 
	что они действительно существуют.
	
	Введём функцию \(g(h) = \Pr_{\infty}(p_{0}(\bm{\xi}) > 
	hp_{\infty}(\bm{\xi}))\), где \(\bm{\xi} = (\xi_{1}, \ldots, \xi_{N})\). 
	Что мы можем сказать про эту функцию? На самом деле много: она не 
	возрастает, она непрерывна справа, \(g(h) = 1\) при \(h < 0\) и \(g(h) \to 
	0\) при \(h \to \infty\). Далее, заметим, что
	\begin{equation}
		g(h) = \int\limits_{\left\{\mathbf{x} \in 
		\R^{N}\,:\,\frac{p_{0}(\mathbf{x})}{p_{\infty}(\mathbf{x})} > 
		h\right\}} p_{\infty}(\mathbf{x})\mu(\diff \mathbf{x}).
	\end{equation}
	
	Для \(a \in (0, 1)\) положим \(h_{a}^{*}\), равным минимальному \(h\), для 
	которого выполнено \(g(h) \leq a \leq g(h - 0)\). Далее, положим
	\begin{equation}
		\lambda_{a}^{*} = \frac{a - g(h_{a}^{*})}{g(h_{a}^{*} - 0) - 
		g(h_{a}^{*})}.
	\end{equation}
	
	Теперь покажем, что с такими значениями \(\E_{\infty}[\phi^{*}] = a\). 
	Действительно,
	\begin{align*}
		\E_{\infty}[\phi^{*}] &= \int\limits_{\R^{N}} \phi^{*}(\mathbf{x}) 
		p_{\infty}(\mathbf{x})\mu(\diff \mathbf{x}) = \int\limits_{\left 
		\{\mathbf{x} \in \R^{N}\,:\,\frac{p_{0}(\mathbf{x})} {p_{\infty} 
		(\mathbf{x})} \geq h_{a}^{*}\right\}} \phi^{*}(\mathbf{x}) 
		p_{\infty}(\mathbf{x})\mu(\diff \mathbf{x}) = \\
		&= \int\limits_{\left \{\mathbf{x} \in \R^{N}\,:\,\frac{p_{0} 
		(\mathbf{x})} {p_{\infty}(\mathbf{x})} > h_{a}^{*}\right\}} 
		p_{\infty}(\mathbf{x})\mu(\diff \mathbf{x}) + 
		\lambda_{a}^{*}\int\limits_{\left\{\mathbf{x} \in 
		\R^{N}\,:\,\frac{p_{0} (\mathbf{x})} {p_{\infty}(\mathbf{x})} = 
		h_{a}^{*}\right\}} p_{\infty}(\mathbf{x})\mu(\diff \mathbf{x}) = \\
		&= g(h_{a}^{*}) + \frac{a - g(h_{a}^{*})}{g(h_{a}^{*} - 0) - 
		g(h_{a}^{*})}[g(h_{a}^{*} - 0) - g(h_{a}^{*})] = a.
	\end{align*}
	
	Теперь нужно описать граничные случаи. Если \(a = 0\), то ошибок первого 
	рода быть не должно. Как этого достичь? Всегда принимать \(H_{\infty}\). 
	Другими словами, полагаем, что \(\phi^{*}(\mathbf{x}) = 0\) и \(h_{a}^{*} = 
	\infty\). Аналогично, для \(a = 1\) мы должны всегда принимать \(H_{0}\). 
	Следовательно, нужно положить \(h_{a}^{*} = 0\) и \(\lambda_{a}^{*} = 1\).
	
	Тем самым был получен тест \(\phi^{*}\), для которого 
	\(\E_{\infty}[\phi^{*}] = a\) и он является оптимальным в классе 
	\(\Phi_{a}\).
\end{proof}

\begin{remark}
	В случае, когда случайные величины \(\xi_{1}, \ldots, \xi_{N}\) независимы 
	и одинаково распределены, для совместной плотности верна формула 
	\eqref{eq:density-independent}. В этом случае удобно ввести следующие 
	обозначения:
	\[
		\zeta_{k} = \log \frac{f_{0}(x_{k})}{f_{\infty}(x_{k})}  \text{ и } 
		Z_{k} = \log L_{n} = \sum_{k = 1}^{n} \zeta_{k}.
	\]
\end{remark}

Теперь рассмотрим пару примеров применения леммы Неймана-Пирсона.
\begin{example}[Бернуллиевские случайные величины]
	Пусть гипотезы \(H_{0}\) и \(H_{\infty}\) утверждают,  что \(\xi_{1}, 
	\ldots, \xi_{N}\)~--- выборка из распределения Бернулли 
	\(\mathrm{Bern}(p_{0})\) или \(\mathrm{Bern}(p_{\infty})\) соответственно, 
	то есть для любого \(1 \leq k \leq N\)
	\[
		\Pr_{\theta}(\xi_{k} = 1) = p_{\theta},
		\quad
		\Pr_{\theta}(\xi_{k} = 0) = 1 - p_{\theta} \equiv q_{\theta},
		\quad \theta = 0 \text{ или } \infty.
	\]
	
	В таком случае отношение правдоподобия, согласно формуле 
	\eqref{eq:likelihood-ratio}, равно
	\[
		L_{n} = \prod_{k = 1}^{N} \left(\frac{p_{0}}{p_{\infty}}\right)^{x_{k}} 
		\left(\frac{q_{0}}{q_{\infty}}\right)^{1 - x_{k}}.
	\]
	
	Возьмём от этого логарифм:
	\begin{align*}
		Z_{n} &= (x_{1} + \ldots + x_{N})\log\frac{p_{0}}{p_{\infty}} + (N - 
		x_{1} + \ldots + x_{N})\log\frac{q_{0}}{q_{\infty}} = \\
		&= (x_{1} + \ldots + x_{N})\log\frac{p_{0}q_{\infty}}{p_{\infty}q_{0}} 
		+ N\log\frac{q_{0}}{q_{\infty}}.
	\end{align*}
	
	Если ввести обозначение \(X_{N} = x_{1} + \ldots + x_{N}\), то оптимальный 
	критерий \eqref{eq:neyman-pearson-criterion} будет выглядеть так:
	\begin{equation}
		\phi^{*}(x_{1}, \ldots, x_{N}) = 
		\begin{cases}
			1, & X_{N} > h_{a}^{*}, \\
			\lambda_{a}^{*}, & X_{N} = h_{a}^{*}, \\
			0, & X_{N} < h_{a}^{*}, \\
		\end{cases}
	\end{equation}
	где константы \(\lambda_{a}^{*}\) и \(h_{a}^{*}\) находятся из 
	предположения о том, что вероятность ошибки первого рода для этого критерия 
	равна \(a\): \(\E_{\infty}[\phi^{*}] = a\).
\end{example}

Вообще говоря, необходимость обращаться к рандомизированным тестам обычно 
связана с дискретностью распределения. В случае, когда распределения имеют 
плотности \(f_{\theta}(x)\), можно строить и детерменированный тест (так как 
вероятность равенства нулевая). Рассмотрим это на следующем примере.
\begin{example}[Нормальные случайные величины]
	Пусть гипотезы \(H_{0}\) и \(H_{\infty}\) утверждают, что \(\xi_{1}, 
	\ldots, \xi_{n}\)~--- выборка из нормального распределения 
	с параметрами \((\mu_{0}, \sigma^{2})\) и \((\mu_{\infty}, \sigma^{2})\) 
	соответственно. Как известно, плотность нормального распределения 
	\(\mathcal{N}(\mu_{\theta}, \sigma^{2})\) равна
	\[
		f_{\theta}(x) = \frac{1}{\sqrt{2\pi\sigma^{2}}}\exp\left\{-\frac{(x - 
		\mu_{\theta})^{2}}{2\sigma^{2}}\right\}.
	\]
	
	Посчитаем логарифм отношения правдоподобия:
	\begin{align*}
		Z_{n} &= \sum_{k = 1}^{n} \left[\frac{(x_{k} - \mu_{\infty})^{2}} 
		{2\sigma^{2}} - \frac{(x_{k} - \mu_{0})^{2}}{2\sigma^{2}}\right] = 
		\sum_{k = 1}^{n} \frac{(\mu_{0} - \mu_{\infty})(2x_{k} - \mu_{0} - 
		\mu_{\infty})}{2\sigma^{2}} = \\
		&= \frac{\mu_{0} - \mu_{\infty}}{\sigma^{2}}\left(X_{N} - \frac{\mu_{0} 
		- \mu_{\infty}}{2}n\right).
	\end{align*}
	
	Для простоты описания скажем, что \(\mu_{0} = \mu\), \(\mu_{\infty} = 0\). 
	Тогда
	\[
		Z_{n} = \frac{\mu}{\sigma^{2}}\left(X_{n} - \frac{\mu}{2}n\right).
	\]
	
	Следовательно, оптимальный критерий \eqref{eq:neyman-pearson-criterion} 
	выглядит так:
	\[
		\phi^{*}(x_{1}, \ldots, x_{n}) = 
		\begin{cases}
			1, & Z_{n} \geq h, \\
			0, & Z_{n} < h. \\
		\end{cases}
	\]
	
	Немного преобразуем условие, введя обозначение \(H = \sigma^{2}h/\mu\):
	\begin{equation}
		\phi^{*}(x_{1}, \ldots, x_{n}) = 
		\begin{cases}
		1, & X_{n} - \mu n/2 \geq H, \\
		0, & X_{n} - \mu n/2 < H. \\
		\end{cases}
	\end{equation}
	
	Чему равны ошибки первого и второго рода для этого критерия? Для их 
	подсчёта вспомним, что \(\xi_{1}, \ldots, \xi_{n}\)~--- независимые и 
	одинаково распределённые случайные величины. Если верна гипотеза 
	\(H_{\infty}\), то \(\E_{\infty}[\xi_{1} + \ldots + \xi_{n}] = 0\) и 
	\(\D_{\infty}[\xi_{1} + \ldots + \xi_{n}] = \sigma^{2}n\). Следовательно,
	\begin{align*}
		\alpha &= \E_{\infty}[\phi^{*}(\bm{\xi})] = \Pr_{\infty}\left(\sum_{k = 
			1}^{n} \xi_{k} - \frac{\mu n}{2} \geq H\right) = 
		\Pr_{\infty}\left(\frac{\xi_{1} + \ldots + \xi_{n}}{\sigma\sqrt{n}} 
		\geq \frac{H + \mu n/2}{\sigma\sqrt{n}}\right) = \\
		&= 1 - \Phi\left(\frac{H + \mu n/2}{\sigma\sqrt{n}}\right),
	\end{align*}
	где \(\Phi\)~--- функция распределения стандартной нормальной случайной 
	величины. Аналогично считается и \(\beta\), только в данном случае 
	\(\E_{0}[\xi_{1} + \ldots + \xi_{n}] = \mu n\) и 
	\(\D_{0}[\xi_{1} + \ldots + \xi_{n}] = \sigma^{2}n\):
	\begin{align*}
		\beta &= \E_{0}[1 - \phi^{*}(\bm{\xi})] = \Pr_{0}\left(\sum_{k = 
		1}^{n} \xi_{k} - \frac{\mu n}{2} < H\right) = \\
		&= \Pr_{0}\left(\frac{\xi_{1} + \ldots + \xi_{n} - \mu n} 
		{\sigma\sqrt{n}} < \frac{H - \mu n/2}{\sigma\sqrt{n}}\right) = 
		\Phi\left(\frac{H - \mu n/2}{\sigma\sqrt{n}}\right).
	\end{align*}
	
	В итоге мы получаем, что
	\begin{equation}\label{eq:normal-distribution-criterion-error}
		\alpha = 1 - \Phi\left(\frac{H + \mu n/2}{\sigma\sqrt{n}}\right),
		\quad
		\beta = \Phi\left(\frac{H - \mu n/2}{\sigma\sqrt{n}}\right).
	\end{equation}
	
	Далее, пусть \(C_{\gamma}\)~--- это квантиль порядка \(\gamma\) для 
	стандартного нормального распределения, то есть корень уравнения 
	\(\Phi(C_{\gamma}) = \gamma\). Тогда, комбинируя это с 
	\eqref{eq:normal-distribution-criterion-error}, получаем, что
	\begin{equation}
		\frac{H + \mu n/2}{\sigma\sqrt{n}} = C_{1 - \alpha}, 
		\quad
		\frac{H - \mu n/2}{\sigma\sqrt{n}} = C_{\beta}.
	\end{equation}
	
	Теперь несложно получить связь между \((\alpha, \beta)\) и \((n, h)\):
	\begin{equation}\label{eq:neyman-pearson-normal-n}
		(C_{1 - \alpha} - C_{\beta})^{2} = \left(\frac{\mu}{\sigma}\right)^{2}n 
		\implies n = \frac{(C_{1 - \alpha} - C_{\beta})^{2}}{(\mu/\sigma)^{2}}.
	\end{equation}
	\begin{equation}\label{eq:neyman-pearson-normal-H}
		\frac{2H}{\sigma} = \sqrt{n}(C_{1 - \alpha} + C_{\beta}) \implies H = 
		\frac{C_{1 - \alpha}^{2} - C_{\beta}^{2}}{2\mu/\sigma^{2}} \implies 
		h = \frac{C_{1 - \alpha}^{2} - C_{\beta}^{2}}{2}.
	\end{equation}
	
	Следовательно, если мы хотим, чтобы у теста \(\phi^{*}\) были вероятности 
	ошибок первого и второго рода, равные \(\alpha\) и \(\beta\) 
	соответственно, то число наблюдений \(n\) и порог \(h\) будут задаваться 
	формулами \eqref{eq:neyman-pearson-normal-n} и 
	\eqref{eq:neyman-pearson-normal-H} соответственно.
\end{example}

\subsection{Последовательные тесты}\label{sec-3}

Постановка рассмотренной выше задачи различения двух статистических гипотез ($H_{0}$ и 
$H_{\infty}$) предполагала, что решение принимается по заданному числу наблюдений~$N$. При 
этом в~классе $\Phi_{a}=\left\{\varphi\colon \E_{\infty}[\phi] \leq a\right\}$ оптимальный тест 
$\varphi^*$ определялся формулой~\eqref{eq:neyman-pearson-criterion}.

Давайте немного изменим постановку задачи. Пусть
\begin{equation}
	\Phi_{\alpha,\beta} = \bigl\{\phi \colon \E_{\infty}[\phi] \leq \alpha,\,\E_{0}[1 - \phi] \leq 
	\beta\bigr\}.
\end{equation}
То есть \(\Phi_{\alpha, \beta}\) "--- это класс тех тестов $\phi$, для которых вероятности ошибок 
первого и второго рода не превосходят $\alpha$ и $\beta$ соответственно. Ранее мы показали, что для 
гипотез относительно среднего значения в~наблюдениях, подчиняющихся гауссовскому распределению, 
число \(N\) необходимых наблюдений и соответствующий порог \(h\) задавались формулами  
\eqref{eq:neyman-pearson-normal-n} и \eqref{eq:neyman-pearson-normal-H} соответственно. Стоит 
заметить, что в данном случае число наблюдений является \emph{неслучайной} величиной.

Рассмотрим теперь еще одну, новую, постановку задачи, принадлежащую А.~Вальду, а именно задачу
\emph{последовательного} различения гипотез. В~сущности, именно эта задача дала импульс развитию 
теории последовательного анализа и теории оптимальных правил остановки.

Но начнём с определений.
\begin{definition}
	Пусть \(x \in \R^{\infty}\), то есть \(x = (x_{1}, \ldots, x_{n}, \ldots)\) "--- числовая 
	последовательность с \(x_{i} \in \R\). \emph{Борелевской \(\sigma\)-алгеброй в \(\R^{\infty}\)} 
	будем называть минимальную \(\sigma\)-алгебру \(\B(\R^{\infty})\), порождённую множествами вида
	\begin{equation}
		\{x \colon x_{1} \in I_{1}, \ldots, x_{n} \in I_{n}\}, \quad n \geq 1,
	\end{equation}
	где \(I_{1}, \ldots, I_{n}\)~--- это (борелевские) множества из \(\B(\R)\). 
\end{definition}

Будем считать, что на \((\R^{\infty}, \B(\R^{\infty}))\) заданы две вероятностные меры \(\Pr_{0}\) 
и \(\Pr_{\infty}\). Пусть координатно заданные случайные величины \(\xi_{k} = \xi_{k}(x)\), где
\(\xi_k(x) = x_k\), являются независимыми и одинаково распределенными по каждой из мер \(\Pr_0\) и 
\(\Pr_{\infty}\). Дополнительно скажем, что у них есть плотность \(f_{\theta}(x)\).

Предположим, что шаг за шагом мы получаем данные \(x_{1}, x_{2}, \ldots, x_{n}, \ldots\), 
являющиеся наблюдениями над случайными величинами \(\xi_{1}, \xi_{2}, \ldots, \xi_{n}, \ldots\). Мы 
хотим  различить две гипотезы $H_0$ и $H_{\infty}$ о том, какое <<действует>> распределение, 
$\Pr_0$ или $\Pr_{\infty}$, используя \emph{последовательные тесты}, определение которого мы сейчас 
дадим. Но для него нужно знать, что такое марковский момент.

\begin{definition}
	Пусть \((\Omega, \F, \Pr)\) "--- вероятностное пространство с заданной на ней фильтрацией 
	\(\mathbb{F} = (\F_{t})_{t \in T}\), где \(T \subseteq [\,0, +\infty)\). Случайную величину 
	\(\tau\), принимающую значения в \(T \cup \{+\infty\}\), будем называть \emph{марковским 
	моментом} относительно фильтрации \(\mathbb{F}\), если для любого \(t \in T\) событие \(\{\tau 
	\leq t\}\) содержится в 
	\(\F_{t}\). 
\end{definition}
\begin{definition}
	\emph{Последовательным тестом} \(\delta\) будем называть пару \((\tau, \phi)\), где
	\begin{itemize}
		\item \(\tau = \tau(x)\) "--- марковский момент (или момент остановки) относительно потока 
		\(\{\F_{n}, n \geq 1\}\), где \(\F_{n} = \sigma(\xi_{1},\ldots, \xi_{n})\), \(\F_{0} = 
		\{\emptyset, \Omega\}\).
		
		\item \(\phi = \phi(x)\) "--- \(\F_{\tau}\)-измеримая функция со значениями в~\([\,0,1]\), 
		где \(\F_{\tau} = \sigma(\xi_{1}, \ldots, \xi_{\tau})\).
	\end{itemize}
\end{definition}
Момент \(\tau\) интерпретируется как момент прекращения наблюдений с~последующим принятием решения 
\(\phi = \phi(x)\), интерпретируемого как вероятность принятия гипотезы \(H_{0}\), когда 
наблюдениями являются \(x_1, \ldots, x_{\tau}\).

Наиболее важными характеристиками тестов \(\delta = (\tau, \phi)\)
являются средние длительности наблюдений \(\E_{0}[\tau]\) и \(\E_{\infty}[\tau]\) и вероятности 
ошибок первого и второго рода \(\alpha(\phi) = \E_{\infty}[\phi]\) и \(\beta(\phi) = \E_{0}[1 - 
\phi]\).

Пусть \(\alpha\) и \(\beta\)~--- это какие-то два числа из \([\,0, 1]\). Пусть
\[
	\Delta(\alpha, \beta) = \{\delta = (\tau, \phi) \mid \E_{\infty}[\tau] < \infty, \E_{0}[\tau] 
	< \infty, \alpha(\phi) \leq \alpha, \beta(\phi) \leq \beta\}.
\]

Другими словами, \(\Delta(\alpha, \beta)\)~--- это класс последовательных тестов, для которых 
матожидание момента остановки конечно, а вероятности ошибок первого и второго рода ограничены 
сверху \(\alpha\) и \(\beta\) соответственно.

\begin{definition}
	Будем говорить, что тест \(\delta^{*} = (\tau^{*}, \phi^{*})\) \emph{оптимален} в классе 
	\(\Delta(\alpha, \beta)\), если для любого другого теста \(\delta \in \Delta(\alpha, \beta)\)
	\[
		\E_{\infty}[\tau^{*}] \leq \E_{\infty}[\tau] \text{ и } \E_{0}[\tau^{*}] \leq \E_{0}[\tau].
	\]
\end{definition}

А. Вальд установил, что при определённых условиях такой оптимальный тест 
\(\delta^{*}=(\tau^{*},\varphi^{*})\) действительно существует. Этот результат совершенно 
неочевиден, так как тест \(\delta^{*}\)  минимизирует два математических ожидания 
\emph{одновременно}.

Доказательство этого факта весьма трудоёмко, так что ограничимся доказательством того, что 
существует \emph{почти оптимальный} тест (смысл этого выражения будет объяснён позднее). Более 
того, мы упростим задачу, ограничившись только детерминированными решающими функциями \(\phi(x) = 
d(x)\), которые принимают только значения \(0\) и \(1\). Дальнейший анализ покажет, что такое 
ограничение легально.

Следующая лемма важна для доказательства почти оптимальности вальдовского теста "--- она дает 
оценку снизу для \(\E_{\infty}[\tau]\) и \(\E_{0}[\tau]\).
\begin{lemma}
	Пусть \(\delta=(\tau,\varphi)\)~--- последовательный тест с вероятностями ошибок первого и 
	второго рода, равными \(\alpha\) и \(\beta\) соответственно, причём \(0 < \alpha + \beta < 1\). 
	Тогда
	\[
		\E_{\infty}[\tau] \geq \frac{\omega(\alpha, \beta)}{\rho_{\infty}} \text{ и } 
		\E_{0}[\tau] \geq \frac{\omega(\beta, \alpha)}{\rho_{0}},
	\]
	где
	\begin{gather*}
		\omega(x, y) = x\log\frac{x}{1 - y} + (1 - x)\log\frac{1 - x}{y} \\
		\rho_{\infty} = \E_{\infty}\left[\log\frac{f_{\infty}(\xi_{1})}{f_0(\xi_{1})}\right], 
		\quad 
Но для начала докажем одно простое утверждение.
\rho_{0} = \E_{0}\left[\log\frac{f_{0}(\xi_{1})}{f_{\infty}(\xi_{1})}\right].
\end{gather*}
Предполагается, что \(\rho_{0}\) и \(\rho_{\infty}\) конечны.
\end{lemma}

\begin{theorem}[Тождество Вальда]
	Пусть \(\xi_{1}, \ldots, \xi_{n}, \ldots\)~--- последовательность независимых и одинаково 
	распределённых величин, а \(\tau\)~--- это случайная величина, не зависящая от \(\xi_{k}\) и 
	принимающая натуральные значения. Тогда
	\[
		\E\Bigl[\sum_{k = 1}^{\tau} \xi_{k}\Bigr] = \E[\tau]\E[\xi_{1}].
	\]
\end{theorem}
\begin{proof}
	Воспользуемся тем, что \(\tau\) принимает значения в \(\N\) и тем, что она не зависит от 
	\(\xi_{k}\):
	\begin{align*}
		\E\Bigl[\sum_{k = 1}^{\tau} \xi_{k}\Bigr] 
		&= \sum_{n = 1}^{\infty}\E\Bigl[\I_{\tau = n}\sum_{k = 1}^{\tau} \xi_{k}\Bigr]
		= \sum_{n = 1}^{\infty}\E\Bigl[\I_{\tau = n}\sum_{k = 1}^{n} \xi_{k}\Bigr] = \\
		&= \sum_{n = 1}^{\infty}\Pr(\tau = n)\E\Bigl[\sum_{k = 1}^{\tau} \xi_{k}\Bigr]
		= \E[\xi_{1}]\sum_{n = 1}^{\infty} n\Pr(\tau = n) = \E[\xi_{1}]\E[\tau].
	\end{align*}
	Тем самым мы получили желаемое.
\end{proof}

Теперь приступим к доказательству самой леммы.
\begin{proof}
	Пусть \(L_{n}\) "--- это отношение правдоподобия для выборки размера \(n\): \(L_{0} = 1\) и 
	\[
		L_{n} = \frac{f_{0}(x_{1}) f_{0}(x_{2})\ldots f_{0}(x_{n})} 
		{f_{\infty}(x_{1}) f_{\infty}(x_{2}) \ldots f_{\infty}(x_{n})}, \quad n \geq 1.
	\]
	
	Далее, введём следующие обозначения:
	\[
		\zeta_{k} = \log \frac{f_{0}(x_{k})}{f_{\infty}(x_{k})}, 
		\quad
		Z_{n} = \log L_{n} = \sum_{k = 1}^{n} \zeta_{k}.
	\]
	
	Доказывать будем только неравенство для \(\E_{0}[\tau]\) "--- неравенство для 
	\(\E_{\infty}[\tau]\) доказывается аналогично.
	
	Рассмотрим \(\E_{0}[Z_{\tau}]\). По тождеству Вальда
	\begin{equation}\label{z-tau-expectation}
		\E_{0}[Z_{\tau}] = \E_{0}\Bigl[\sum_{k = 1}^{\tau}\zeta_{k}\Bigr] = 
		\E_{0}[\zeta_{1}]\E_{0}[\tau] = \rho_{0}\E_{0}[\tau] \implies \E_{0}[\tau] = 
		\frac{\E_{0}[Z_{\tau}]}{\rho_{0}}.
	\end{equation}
	
	Мы хотим доказать, что
	\begin{equation}\label{eq:wald-expectation-key-ineqality}
		\E_{0}[Z_{\tau}] \geq \omega(\beta, \alpha), \text{ где } \omega(x, y) = x\log\frac{x}{1 - 
		y} + (1 - x)\log\frac{1 - x}{y}.
	\end{equation}
	
	Представим \(\E_{0}[Z_{\tau}]\) в~следующем виде:
	\[
		\E_{0}[Z_{\tau}] 
		= \int\limits_{\{x\colon d(x) = 0\}} Z_{\tau} \diff \Pr_0
		+ \int\limits_{\{x\colon d(x) = 1\}} Z_{\tau} \diff \Pr_0.
	\]
	
	Далее, вспомним неравенство Йенсена: если \(\phi(\cdot)\) "--- выпуклая (вниз) борелевская 
	функция, то
	\begin{equation}\label{eq:jensen-inequality}
		\phi(\E[\xi]) \leq \E[\phi(\xi)].
	\end{equation}
	
	Пользуясь им, получаем, что
	\begin{align}
	\int\limits_{\{x \colon d(x) = 0\}} Z_{\tau} \diff \Pr_0 
	&= \Pr_{0}(d(x) = 0) \int\limits_{\R^{\infty}} Z_{\tau}\,\Pr_0(\diff x\,|\,d(x) = 0) \nonumber 
	\\
	&= \beta \int\limits_{\R^\infty} \log L_{\tau}\,\Pr_0(\diff x\,|\,d(x) = 0) \nonumber \\
	&= -\beta \int\limits_{\R^{\infty}} \log\frac{1}{L_{\tau}}\,\Pr_0(\diff x\,|\,d(x) = 0) 
	\nonumber \\
	&\geq -\beta \log\int\limits_{\R^\infty}\frac1{L_\tau}\,\Pr_0(\diff x\,|\,d(x)=0) \nonumber \\
	&=-\beta \log\biggl[\frac{\int_{\{x\colon 
			d(x)=0\}}\frac{1}{L_{\tau}}\,\Pr_0(dx)}{\Pr_0(d(x)=0)}\biggr].
	\end{align}
\end{proof}
\begin{proof}
	Пусть $L_0=1$ и
	\begin{gather*}
	L_n
	=\frac
	{f_{0}(x_1)\cdots f_{0}(x_n)}
	{f_{\infty}(x_1)\cdots f_{\infty}(x_n)},
	\qquad
	n\ge1,\\
	Z_n=\log L_n,
	\quad
	\zeta_n=\log\frac{f_0(x_n)}{f_{\infty}(x_n)}.
	\end{gather*}
	Будем доказывать неравенство~\eqref{38}. Неравенство~\eqref{37}
	доказывается аналогично.
	
	Рассмотрим
	\begin{equation}
	\label{42}%
	\E_0Z_{\tau}=\E_0\sum_{k=1}^{\tau}\zeta_k.
	\end{equation}
	Согласно \textit{тождеству Вальда}, если
	\[
	\E_0|\zeta_k|<\infty
	\quad\text{и}\quad
	\E_0\tau<\infty,
	\]
	то
	\begin{equation}
	\label{43}%
	\E_0Z_{\tau}=\rho_0\,\E_0\tau.
	\end{equation}
	(Доказательство тождества Вальда мы проведем на лекциях,
	посвященных теории мартингалов.) Используя неравенство Иенсена
	\[
	\log\E_0\zeta_1\ge\E_0\log\zeta_1\;(=\rho_0)
	\]
	и естественное предположение, что меры $\Pr_0$ и $\Pr_{\infty}$ не
	совпадают, можно вывести, что $\rho_0>0$.
	
	Если $\rho_0=\infty$, то~\eqref{38} очевидно. Так что с~самого
	начала можно считать, что $\rho_0<\infty$, и тогда из~\eqref{43}
	заключаем, что
	\begin{equation}
	\label{44}%
	\E_0\tau=\frac{\E_0Z_{\tau}}{\rho_0}.
	\end{equation}
	
	Тем самым для доказательства неравенства~\eqref{38} надо показать,
	что
	\begin{equation}
	\label{45}%
	\E_0Z_{\tau}\ge\omega(\beta,\alpha).
	\end{equation}
	Представим $\E_0Z_{\tau}$ в~следующем виде:
	\[
	\E_0Z_{\tau}
	=\int_{\{x\colon d(x)=0\}}Z_{\tau}\,d\Pr_0
	+\int_{\{x\colon d(x)=1\}}Z_{\tau}\,d\Pr_0.
	\]
	Снова используя неравенство Иенсена
	($\log\E_0\xi\ge\E_0\log\xi$), находим, что
	\begin{align}
	\label{46}%
	\int_{\{x\colon d(x)=0\}}Z_{\tau}\,d\Pr_0 &
	=\Pr_0(d(x)=0) \int_{\R^{\infty}}Z_{\tau}\,\Pr_0(dx\mid d(x)=0)\notag\\
	&= \beta\int_{\R^\infty}\log L_\tau\,\Pr_0(dx\mid d(x)=0)\notag\\
	&= -\beta\int_{\R^{\infty}}\log\frac{1}{L_{\tau}}\,\Pr_0(dx\mid 
	d(x)=0)\notag\\
	& \ge-\beta\log\int_{\R^\infty}\frac1{L_\tau}\,\Pr_0(dx\mid d(x)=0)\notag\\
	&=-\beta\log\biggl[\frac{\int_{\{x\colon 
	d(x)=0\}}\frac{1}{L_{\tau}}\,\Pr_0(dx)}{\Pr_0(d(x)=0)}\biggr].
	\end{align}
	Покажем, что
	\begin{equation}
	\label{47}%
	\int_{\{x\colon d(x)=0\}}
	\frac{1}{L_{\tau}}
	\,\Pr_0(dx)\le\Pr_{\infty}(d(x)=0)
	=1-\alpha.
	\end{equation}
	Имеем
	\[
	\int_{\{x\colon d(x)=0\}}\frac{1}{L_{\tau}}\,\Pr_0(dx)
	=\sum_{n=1}^{\infty}\int_{A_n}\frac{1}{L_{\tau}}\,\Pr_0(dx),
	\]
	где $A_n=\{x\colon d(x)=0\}\cap\{\tau(x)=n\}$.
	
	Здесь, как нетрудно видеть,
	\begin{align}
	\int_{A_n}\frac{1}{L_{\tau}}\,\Pr_0(dx)
	&
	=\int_{A_n}\frac{1}{L_n}\,\Pr_0(dx)
	=\int_{A_n}
	\frac
	{f_{\infty}(x_1)\cdots f_{\infty}(x_n)}
	{f_{0}(x_1)\cdots f_{0}(x_n)}
	\,\Pr_0(dx)\notag\\
	&
	=\int_{A_n}\,\Pr_{\infty}(dx)
	=\Pr_{\infty}(A_n)
	=\Pr_{\infty}(\{d=0\}\cap\{\tau=n\}).\notag
	\end{align}
	Следовательно,
	\[
	\int_{\{x\colon d(x)=0\}}\frac{1}{L_{\tau}}\,\Pr_0(dx)
	=\Pr_{\infty}(d=0)
	=\E_{\infty}(1-\alpha)
	=1-\alpha.
	\]
	Тем самым~\eqref{47} установлено, и в~силу~\eqref{46}
	\begin{equation}
	\label{48}%
	\int_{\{x\colon d(x)=0\}}Z_{\tau}\,d\Pr_0
	\ge-\beta\log\frac{1-\alpha}{\beta}
	=\beta\log\frac{\beta}{1-\alpha}.
	\end{equation}
	Аналогично устанавливается, что
	\begin{equation}
	\label{49}%
	\int_{\{x\colon d(x)=1\}}Z_{\tau}\,d\Pr_0
	\ge(1-\beta)\log\frac{1-\beta}{\alpha}.
	\end{equation}
	Из~\eqref{48} и~\eqref{49} получаем, что
	\begin{equation}
	\label{50}%
	\E_0Z_{\tau}
	\ge\beta\log\frac{\beta}{1-\alpha}+(1-\beta)\log\frac{1-\beta}{\alpha}
	=\omega(\beta,\alpha).
	\end{equation}
	
	Аналогично~\eqref{45},
	\begin{equation}
	\label{51}%
	\E_{\infty}Z_{\tau}\ge\omega(\alpha,\beta).
	\end{equation}
	
	Из~\eqref{44} и~\eqref{50} получаем требуемое
	неравенство~\eqref{38}. Из~\eqref{50} и равенства
	$\E_{\infty}\tau=\rho_{\infty}^{-1}\E_{\infty}Z_{\tau}$ получаем
	неравенство~\eqref{37}. Лемма доказана.
\end{proof}

\textbf{3.} Ценность неравенств~\eqref{37} и~\eqref{38} ясна: если
суметь (для заданных $\alpha$ и~$\beta$) построить тест
$\delta^*=(\tau^*,d^*)$, для которого $\E_{\infty}\tau^*$ и
$\E_0\tau^*$ задаются правыми частями в~\eqref{37} и~\eqref{38},
то этот тест будет оптимальным. Мы увидим далее, что в~случае
\textit{непрерывного} времени (в~задаче различения гипотез
относительно среднего значения броуновского движения) такой тест
действительно можно построить. Но ясно также, что если суметь
построить тест $\tilde{\delta}=(\widetilde{\tau},\tilde{d})$,
у~которого значения $\E_{\infty}\widetilde{\tau}$ и
$\E_0\widetilde{\tau}$ близки к~правым частям в~\eqref{37} и
\eqref{38}, то это будет говорить о~том, что такой тест ``почти
оптимален''. Займемся конструкцией таких тестов.

Выберем две константы $A<0$ и $B>0$ и положим
\[
\tau_{A,B}=\inf\{n\ge1\colon Z_n=B\mbox{ или } Z_n=A\},
\]
т.\,е.
\[
\tau_{A,B}=\inf\{n\ge1\colon Z_n\not\in(A,B)\},
\]
где
\[
Z_n
=\log L_n
=\sum_{k=1}^n\zeta_k,
\quad
\zeta_k
=\log\frac{f_0(x_k)}{f_{\infty}(x_k)}.
\]
Введем также решающее правило
\[
d_{A,B}=
\left\{%
\begin{array}{ll}
1, & \hbox{если $Z_{\tau_{A,B}}\ge B$,} \\
0, & \hbox{если $Z_{\tau_{A,B}}\le A$.} \\
\end{array}%
\right.
\]
Тест $\delta_{A,B}=(\tau_{A,B},d_{A,B})$ был предложен А.~Вальдом
и называется \textit{последовательным критерием отношений
	вероятностей} (см. на рис.~\ref{fig2} схематическое изображение
траектории процесса $Z_n,n\ge1$ и момента остановки~$\tau_{A,B}$).

При заданных $A$ и $B$ найдем соответствующие ошибки первого и
второго рода
\begin{align*}
\alpha
&
=\E_{\infty}d_{A,B}
=\Pr_{\infty}(Z_{\tau_{A,B}}\ge B)
\intertext{и}%
\beta
&
=\E_{0}(1-d_{A,B})
=\Pr_{0}(Z_{\tau_{A,B}}\le A),
\end{align*}
а также найдем $\E_{\infty}\tau_{A,B}$ и $\E_0\tau_{A,B}$.

Для отыскания $\E_{\theta}\tau_{A,B}$ ($\theta=0,\infty$) опять
воспользуемся уже упомянутым (первым) тождеством Вальда, утверждающим, что если
\begin{equation}
\label{52}%
\E_{\theta}|\zeta_1|<\infty
\quad\text{и}\quad
\E_{\theta}\tau<\infty,
\end{equation}
то для независимых и одинаково распределенных случайных величин
$\zeta_1,\zeta_2,\ldots$ справедливо равенство (ср. с~\eqref{43})
\begin{equation}
\label{53}%
\E_{\theta}\left(\zeta_1+\ldots+\zeta_{\tau}\right)
=\E_{\theta}\zeta_1\cdot\E_{\theta}\tau.
\end{equation}

Применительно к~$\zeta_k=\log(f_0(x_k)/f_{\infty}(x_k))$ и
$\tau=\tau_{A,B}$ из~\eqref{53} находим, пренебрегая эффектом
перескока через границу, что
\begin{align}
\label{54}%
\E_0\tau_{A,B}
&
=\frac{\E_0Z_{\tau_{A,B}}}{\E_0\log\frac{f_0(\xi_1)}{f_{\infty}(\xi_1)}}
\simeq
\frac
{\Pr_0(d_{A,B}=1)B+\Pr_0(d_{A,B}=0)A}{\rho_0}\notag\\
&
=\frac{B\E_0d_{A,B}+A\E_0(1-d_{A,B})}{\rho_0}
=\frac{B(1-\beta)+A\beta}{\rho_0}.
\end{align}
Аналогично,
\begin{align}
\label{55}%
\E_{\infty}\tau_{A,B}
&
=\frac{\E_{\infty}Z_{\tau_{A,B}}}
{\E_{\infty}\log(f_0(\xi_1)/f_{\infty}(\xi_1))}
=\frac{\E_{\infty}Z_{\tau_{A,B}}}{-\rho_{\infty}}\notag\\
&
\simeq\frac{B\E_{\infty}d_{A,B}+A\E_{\infty}(1-d_{A,B})}{-\rho_{\infty}}
=\frac{\alpha B+A(1-\alpha)}{-\rho_{\infty}}.
\end{align}

Займемся отысканием формул связи ошибок $(\alpha,\beta)$
с~порогами $(A,B)$.

С этой целью обратимся к~так называемому \textit{фундаментальному
	тождеству Вальда}.

Пусть
\begin{equation}
\label{56}%
g_{\theta}(\lambda)
=\E_{\theta}e^{\lambda\zeta_1}=\E_{\theta}
\left[\frac{f_0(\xi_1)}{f_{\infty}(\xi_1)}\right]^{\lambda},
\qquad
\lambda\in\R.
\end{equation}
Понятно, что для независимых и одинаково распределенных величин
$\zeta_1,\zeta_2,\ldots$ выполнено равенство
\[
\E_{\theta}\exp\{\lambda(\zeta_1+\ldots+\zeta_n)\}
=[g_{\theta}(\lambda)]^n,
\]
т.\,е.
\begin{equation}
\label{57}%
\E_{\theta}\frac{e^{\lambda Z_n}}{[g_{\theta}(\lambda)]^n}=1,
\end{equation}
или
\begin{equation}
\label{58}
\E_{\theta}\exp\{\lambda Z_n-n\log g_{\theta}(\lambda)\}=1.
\end{equation}
(Предполагается, что все выражения, приведенные здесь и ниже
в~\eqref{59}, определены и конечны по крайней мере для ``нужных''
нам в~дальнейшем значений~$\lambda$.)

\textbf{Фундаментальное тождество Вальда.} Справедливо следующее
утверждение: для марковского момента~$\tau$ и~$\lambda\in\R$
\begin{equation}
\label{59}%
\E_{\theta}\frac{e^{\lambda
		Z_{\tau}}}{[g_{\theta}(\lambda)]^{\tau}}=1,
\end{equation}
или
\begin{equation}
\label{60}
\E_{\theta}
\exp\{\lambda Z_\tau-\tau\log g_\theta(\lambda)\}=1.
\end{equation}
(Доказательство этого фундаментального тождества будет дано
в~лекциях, посвященных теории мартингалов.)

Предположим сейчас, что найдется такое ``нужное значение''
$\lambda=\lambda_{\theta}$, что
\begin{equation}
\label{61}%
g_{\theta}(\lambda_{\theta})=1.
\end{equation}

Очевидным образом~\eqref{61} выполняется при $\lambda_{\theta}=0$.
Но есть и нетривиальные $\lambda_{\theta}$, для которых~\eqref{61}
выполнено.

Если $\theta=0$, то
\[
g_0(\lambda)=\E_0\left[\frac{f_0(\xi_1)}{f_{\infty}(\xi_1)}\right]^{\lambda},
\]
и при $\lambda=\lambda_0=-1$ видим, что
\begin{equation}
\label{62}%
g_0(\lambda_0)=1.
\end{equation}
Аналогично,
\[
g_{\infty}(\lambda)=\E_{\infty}\left[\frac{f_0(\xi_1)}{f_{\infty}(\xi_1)}\right]^{\lambda}
\]
и при $\lambda=\lambda_{\infty}=1$
\begin{equation}
\label{63}%
g_{\infty}(\lambda_{\infty})=1.
\end{equation}

Из~\eqref{60} и~\eqref{62} находим, что
\begin{equation}
\label{64}%
\E_0\exp\{-Z_{\tau_{A,B}}\}=1,
\end{equation}
и из~\eqref{60},~\eqref{63} видим, что
\begin{equation}
\label{65}%
\E_{\infty}\exp\{Z_{\tau_{A,B}}\}=1.
\end{equation}
Таким образом, из~\eqref{64}, снова пренебрегая эффектом перескока
через границу, находим, что
\[
e^{-B}\Pr_0\left( d_{A,B}=1\right)+e^{-A}\Pr_0\left( d_{A,B}=0\right)\simeq1,
\]
или
\[
e^{-B}(1-\beta)+e^{-A}\beta\simeq1,
\]
откуда
\begin{equation}
\label{66}%
\beta
=\Pr_0\left( d_{A,B}
=0\right)\simeq\frac{e^B-1}{e^{B-A}-1}.
\end{equation}
Аналогично, из~\eqref{65}
\[
e^B\Pr_{\infty}\left( d_{A,B}
=1\right)+e^A\Pr_{\infty}\left( d_{A,B}=0\right)
\simeq1,
\]
или
\[
e^B\alpha+e^A(1-\alpha)\simeq1,
\]
откуда
\begin{equation}
\label{67}%
\alpha
=\Pr_{\infty}\left( d_{A,B}=1\right)
\simeq\frac{e^{-A}-1}{e^{B-A}-1}.
\end{equation}
Из~\eqref{66} и~\eqref{67} видим, что для (приближенного)
получения пары ошибок $(\alpha,\beta)$ надо просто положить
(в~предположении, что $0<\alpha+\beta<1$)
\begin{equation}
\label{68}%
A=\log\frac{\beta}{1-\alpha},
\quad
B=\log\frac{1-\beta}{\alpha}.
\end{equation}
Вернемся к~формулам~\eqref{54} и~\eqref{55}. Из~\eqref{54}
и~\eqref{68} имеем
\begin{align*}
\E_0\tau_{A,B}
&
\simeq\frac{B(1-\beta)+A\beta}{\rho_0}\\
&
\simeq\frac{(1-\beta)\log\frac{1-\beta}{\alpha}+\beta\log\frac{\beta}{1-\alpha}}{\rho_0}
=\frac{\omega(\beta,\alpha)}{\rho_0}.
\end{align*}
Из~\eqref{55}
\begin{align*}
\E_{\infty}\tau_{A,B}
&
\simeq\frac{\alpha B+A(1-\alpha)}{-\rho_{\infty}}
\simeq\frac{\alpha\log\frac{1-\beta}{\alpha}+(1-\alpha)\log\frac{\beta}{1-\alpha}}{-\rho_{\infty}}\\
&
=\frac{\alpha\log\frac{\alpha}{1-\beta}+(1-\alpha)\log\frac{1-\alpha}{\beta}}{\rho_{\infty}}
=\frac{\omega(\alpha,\beta)}{\rho_{\infty}}.
\end{align*}
Итак, в~предположении пренебрежения эффектом перескока через
границу
\begin{equation}
\label{69}%
\E_0\tau_{A,B}\simeq\frac{\omega(\beta,\alpha)}{\rho_0}
\end{equation}
и
\begin{equation}
\label{70}%
\E_{\infty}\tau_{A,B}\simeq\frac{\omega(\alpha,\beta)}{\rho_{\infty}}.
\end{equation}
Сравнивая эти формулы с~\eqref{38} и~\eqref{37}, видим, что
в~классе тестов $(\tau,d)$ таких, что $\E_0\tau<\infty$,
$\E_{\infty}\tau<\infty$ и $\Pr_0(d=0)\le\alpha$,
$\Pr_{\infty}(d=1)\le\beta$, тест $(\tau_{A,B},d_{A,B})$ является
``почти оптимальным'' (с~точностью до пренебрежения эффектом
перескока процессом $(Z_n)_{n\ge1}$ порогов $A$ и~$B$, которые
подсчитываются по $\alpha$ и~$\beta$ с~помощью формул~\eqref{68}).

